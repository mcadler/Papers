\documentclass[subfigmatrix]{aiaa-tc}% add 'draft' option to show overfull boxes

\usepackage{graphicx,times}
\usepackage{enumerate}\usepackage{amsfonts}
\usepackage{amsmath}
%\title{Progress in High Speed Shock Wave/Turbulent
%Boundary Layer Interactions}
\title{Progress in Shock Wave/Boundary Layer Interactions}
\author{
\begin{tabular}{c}
{\bf D.V. Gaitonde}\thanks{John Glenn Professor, Fellow AIAA, gaitonde.3@osu.edu} \\ 
{\itshape Mech. and Aerosp. Eng. Dept.} \\
{\itshape The Ohio State University}  \\ 
{\itshape Columbus, OH} 
\end{tabular}
}
\newcommand{\baselinedirectory}{??}
\newcommand{\QPDDIRECTORY}{}
\newcommand{\airfoilbase}{}

\setcounter{secnumdepth}{3}
\begin{document}

\maketitle

\section*{Abstract} 
Recent advances in shock wave boundary layer interaction research are
reviewed in four areas: i) understanding low frequency unsteadiness,
ii) heat transfer prediction capability, iii) phenomena in complex
(multi-shock boundary layer) interactions and iv) flow control
techniques.  Substantial success has been achieved in describing the
phenomenology of low frequency unsteadiness, including correlations
and coherent structures in the separation bubble, through
complementary experimental and numerical studies on nominally
\mbox{2-D} interactions.  These observations have been parlayed to
propose underlying mechanisms based on oscillation, amplification and
upstream boundary layer effects.  For heat transfer prediction
capability, systematic studies conducted under the auspices of {\em
  AFOSR} and {\em RTO-AVT} activities have shown that for axisymmetric
laminar situations, heat transfer rates can be measured and in some
cases be predicted reasonably accurately even in the presence of
high-temperature effects. Efforts have quantified uncertainty of
Reynolds averaged turbulence models, and hybrid methods have been
developed to at least partially address deficiencies.  Progress in
complex interactions encompass two of the major phenomena affected by
{\em SBLI} in scramjet flowpaths: unstart and mode transition from
ramjet (dual mode) to scramjet.  Control studies have attempted to
leverage the better understanding of the fundamental phenomena with
passive and active techniques, the latter exploiting the superior
properties of newer actuators. Of interest are not only reduction in
separation and surface loads, but also the spectral content. Finally,
{\em SBLI} studies have benefited handsomely from successful ground
and flight test campaigns associated with the {\em HIFiRE-1} and {\em
  HIFiRE-2} campaigns, results from which which are woven into the
discussion, as are limitations in current capability and
understanding.


\bibliographystyle{unsrt}
\bibliography{./MASTER}






\end{document}

% FIGURE TEMPLATE
%\begin{figure}
%\begin{subfigmatrix}{2}
%  \subfigure[Spatial evolution of pressure]
%%{\includegraphics[width=2.75in]{../../../QPD/PULSEVAR/1D1/WAVE2.png}}
%{\includegraphics[width=2.75in]{\baselinedirectory/../WAVE2lnk.png}}
%%\subfigure[Spatial evolution of temperature]
%%%{\includegraphics[width=3in]{../../../QPD/PULSEVAR/1D1/WAVE1.png}}
%%{\includegraphics[width=3in]{WAVE1lnk.png}}
%\subfigure[Time history of signals at probes]
%%{\includegraphics[width=3in]{../../../QPD/PULSEVAR/1D1/APOINT1.png}}
%{\includegraphics[width=2.75in]{\baselinedirectory/../APOINT1lnk.png}}
%\end{subfigmatrix}
%\caption{Evolution of signal on start and end of long thermal surface
%  heating pulse}\label{startstop}
%\end{figure}
